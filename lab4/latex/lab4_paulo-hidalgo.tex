%package list
\documentclass{article}
\usepackage[top=3cm, bottom=3cm, outer=3cm, inner=3cm]{geometry}
\usepackage{multicol}
\usepackage{graphicx}
\usepackage{url}
%\usepackage{cite}
\usepackage{hyperref}
\usepackage{array}
%\usepackage{multicol}
\newcolumntype{x}[1]{>{\centering\arraybackslash\hspace{0pt}}p{#1}}
\usepackage{natbib}
\usepackage{pdfpages}
\usepackage{multirow}
\usepackage[normalem]{ulem}
\useunder{\uline}{\ul}{}
\usepackage{svg}
\usepackage{xcolor}
\usepackage{listings}
\lstdefinestyle{ascii-tree}{
    literate={├}{|}1 {─}{--}1 {└}{+}1 
  }
\lstset{basicstyle=\ttfamily,
  showstringspaces=false,
  commentstyle=\color{red},
  keywordstyle=\color{blue}
}
%\usepackage{booktabs}
\usepackage{caption}
\usepackage{subcaption}
\usepackage{float}
\usepackage{array}

\newcolumntype{M}[1]{>{\centering\arraybackslash}m{#1}}
\newcolumntype{N}{@{}m{0pt}@{}}


%%%%%%%%%%%%%%%%%%%%%%%%%%%%%%%%%%%%%%%%%%%%%%%%%%%%%%%%%%%%%%%%%%%%%%%%%%%%
%%%%%%%%%%%%%%%%%%%%%%%%%%%%%%%%%%%%%%%%%%%%%%%%%%%%%%%%%%%%%%%%%%%%%%%%%%%%
\newcommand{\itemEmail}{phidalgo@unsa.edu.pe}
\newcommand{\itemStudent}{Paulo Andre Hidalgo Chinchay}
\newcommand{\itemCourse}{Programación web 2}
\newcommand{\itemCourseCode}{20223011}
\newcommand{\itemSemester}{III}
\newcommand{\itemUniversity}{Universidad Nacional de San Agustín de Arequipa}
\newcommand{\itemFaculty}{Facultad de Ingeniería de Producción y Servicios}
\newcommand{\itemDepartment}{Departamento Académico de Ingeniería de Sistemas e Informática}
\newcommand{\itemSchool}{Escuela Profesional de Ingeniería de Sistemas}
\newcommand{\itemAcademic}{2023 - A}
\newcommand{\itemInput}{Del 2 Junio 2023}
\newcommand{\itemOutput}{Al 5 Junio 2023}
\newcommand{\itemPracticeNumber}{03}
\newcommand{\itemTheme}{NodeJS con express}
%%%%%%%%%%%%%%%%%%%%%%%%%%%%%%%%%%%%%%%%%%%%%%%%%%%%%%%%%%%%%%%%%%%%%%%%%%%%
%%%%%%%%%%%%%%%%%%%%%%%%%%%%%%%%%%%%%%%%%%%%%%%%%%%%%%%%%%%%%%%%%%%%%%%%%%%%

\usepackage[english,spanish]{babel}
\usepackage[utf8]{inputenc}
\AtBeginDocument{\selectlanguage{spanish}}
\renewcommand{\figurename}{Figura}
\renewcommand{\refname}{Referencias}
\renewcommand{\tablename}{Tabla} %esto no funciona cuando se usa babel
\AtBeginDocument{%
	\renewcommand\tablename{Tabla}
}

\usepackage{fancyhdr}
\pagestyle{fancy}
\fancyhf{}
\setlength{\headheight}{30pt}
\renewcommand{\headrulewidth}{1pt}
\renewcommand{\footrulewidth}{1pt}
\fancyhead[L]{\raisebox{-0.2\height}{\includegraphics[width=3cm]{img/logo_episunsa.png}}}
\fancyhead[C]{\fontsize{7}{7}\selectfont	\itemUniversity \\ \itemFaculty \\ \itemDepartment \\ \itemSchool \\ \textbf{\itemCourse}}
\fancyhead[R]{\raisebox{-0.2\height}{\includegraphics[width=1.2cm]{img/logo_abet}}}
\fancyfoot[L]{Estudiante Paulo Hidalgo Chinchay}
\fancyfoot[C]{\itemCourse}
\fancyfoot[R]{Página \thepage}

% para el codigo fuente
\usepackage{listings}
\usepackage{color, colortbl}
\definecolor{dkgreen}{rgb}{0,0.6,0}
\definecolor{gray}{rgb}{0.5,0.5,0.5}
\definecolor{mauve}{rgb}{0.58,0,0.82}
\definecolor{codebackground}{rgb}{0.95, 0.95, 0.92}
\definecolor{tablebackground}{rgb}{0.8, 0, 0}

\lstset{frame=tb,
	language=bash,
	aboveskip=3mm,
	belowskip=3mm,
	showstringspaces=false,
	columns=flexible,
	basicstyle={\small\ttfamily},
	numbers=none,
	numberstyle=\tiny\color{gray},
	keywordstyle=\color{blue},
	commentstyle=\color{dkgreen},
	stringstyle=\color{mauve},
	breaklines=true,
	breakatwhitespace=true,
	tabsize=3,
	backgroundcolor= \color{codebackground},
}

\begin{document}
	
	\vspace*{10px}
	
	\begin{center}	
		\fontsize{17}{17} \textbf{ Informe de Laboratorio \itemPracticeNumber}
	\end{center}
	\centerline{\textbf{\Large Tema: \itemTheme}}
	%\vspace*{0.5cm}	

	\begin{flushright}
		\begin{tabular}{|M{2.5cm}|N|}
			\hline 
			\rowcolor{tablebackground}
			\color{white} \textbf{Nota}  \\
			\hline 
			     \\[30pt]
			\hline 			
		\end{tabular}
	\end{flushright}	

	\begin{table}[H]
		\begin{tabular}{|x{4.7cm}|x{4.8cm}|x{4.8cm}|}
			\hline 
			\rowcolor{tablebackground}
			\color{white} \textbf{Estudiante} & \color{white}\textbf{Escuela}  & \color{white}\textbf{Asignatura}   \\
			\hline 
			{\itemStudent \par \itemEmail} & \itemSchool & {\itemCourse \par Semestre: \itemSemester \par Código: \itemCourseCode}     \\
			\hline 			
		\end{tabular}
	\end{table}		
	
	\begin{table}[H]
		\begin{tabular}{|x{4.7cm}|x{4.8cm}|x{4.8cm}|}
			\hline 
			\rowcolor{tablebackground}
			\color{white}\textbf{Laboratorio} & \color{white}\textbf{Tema}  & \color{white}\textbf{Duración}   \\
			\hline 
			\itemPracticeNumber & \itemTheme & 04 horas   \\
			\hline 
		\end{tabular}
	\end{table}
	
	\begin{table}[H]
		\begin{tabular}{|x{4.7cm}|x{4.8cm}|x{4.8cm}|}
			\hline 
			\rowcolor{tablebackground}
			\color{white}\textbf{Semestre académico} & \color{white}\textbf{Fecha de inicio}  & \color{white}\textbf{Fecha de entrega}   \\
			\hline 
			\itemAcademic & \itemInput &  \itemOutput  \\
			\hline 
		\end{tabular}
	\end{table}
	
	\section{Tarea}
	\begin{itemize}		
		\item  Cree una aplicación NodeJS con express, para administrar una agenda personal.
		\subitem Crear evento: fecha y hora. (Si ya existe el archivo no debería ingresar el evento)(La primera línea es el título del evento, las demás líneas son la descripción del evento.)
		\subitem Editar evento. (Se muestran el archivo  donde esta el detalle del evento)
		\subitem Eliminar evento.
		\subitem Ver eventos. Utilizar el formato árbol especificado anteriormente, donde debería incluirse sólo el título del evento.
		\item Interfaz que trabaja todo un CRUD en una sola vista. (Se pueden usar ventanas emergentes)
	\end{itemize}
		
	\section{Equipos, materiales y temas utilizados}
	\begin{itemize}
		\item Sistema Operativo Ubuntu GNU Linux 23 lunar 64 bits Kernell 6.2.
		\item Sistema Operativo Windows 11 pro versión 22H2 de 64 bits.
		\item VIM 9.0.
		\item Git 2.39.2.
		\item Visual Studio Code 1.78.2.
		\item Cuenta en GitHub con el correo institucional.
		\item NodeJS 18.1.0
		\item Ejercicios resueltos.
		\item \url{https://github.com/rescobedoq/pw2/tree/main/labs/lab03}
		\item Saber usar File System de NodeJS.
		\item \url{https://www.w3schools.com/nodejs/nodejs_filesystem.asp}
	\end{itemize}
	
	\section{URL de Repositorio Github}
	\begin{itemize}
		\item URL del Repositorio GitHub para clonar o recuperar.
		\item \url{https://github.com/PauloUNSA/pw2-lab-c-23a.git}
		\item URL para el laboratorio 03 en el Repositorio GitHub.
		\item \url{https://github.com/PauloUNSA/pw2-lab-c-23a/tree/main/lab3}
	\end{itemize}
	
	\section{App Agenda en NodeJS con express}
	%def JS
	\lstdefinelanguage{JS}{
  keywords={typeof, new, true, false, catch, function, return, null, catch,
  switch, var, if, in, while, do, else, case, break},
  keywordstyle=\color{blue}\bfseries,
  ndkeywords={class, export, boolean, throw, implements, import, this},
  ndkeywordstyle=\color{darkgray}\bfseries,
  identifierstyle=\color{black},
  sensitive=false,
  comment=[l]{//},
  morecomment=[s]{/*}{*/},
  commentstyle=\color{purple}\ttfamily,
  stringstyle=\color{red}\ttfamily
}
%def CSS
\lstdefinelanguage{CSS}{
  keywords={color,background,margin,padding,font,weight,display,position,top,left,right,bottom,list,style,border,size,white,space,min,width},
  sensitive=true,
  morecomment=[l]{//},
  morecomment=[s]{/*}{*/},
  morestring=[b]',
  morestring=[b]",
  alsoletter={:},
  alsodigit={-}
}
	\subsection{Configurar espacio de trabajo}
	\begin{itemize}	
		\item Con el comando npm se instalaron los siguientes frameworks:
		\item express, body-parser, markdown-it y cors.
		\item Ademas se crearon los archivos index.js e index.html.
		de prueba para ver si funcionaba correctamente el entorno.
	\end{itemize}	
	
	\begin{lstlisting}[language=bash,caption={Crear subcarpeta para instalar frameworks}][H]
		$ mkdir -p /express/
	\end{lstlisting}
	\begin{lstlisting}[language=bash,caption={Instalar frameworks dentro de express}][H]
		$ cd /express
		$ npm install express body-parser markdown-it cors
	\end{lstlisting}	
	%%\lstinputlisting[language=JS, caption={Index.js de prueba},numbers=left,]{src/indexjs01.js}
	%%\lstinputlisting[language=JS, caption={Index.html de prueba},numbers=left,]{src/index01.html}
	\begin{figure}[H]
		\centering
		\includegraphics[width=0.8\textwidth,keepaspectratio]{img/localhost01.png}
	\end{figure}
	\begin{figure}[H]
		\centering
		\includegraphics[width=0.8\textwidth,keepaspectratio]{img/localhost_agenda.png}
	\end{figure}
	\begin{itemize}	
		\item Como se observa funciona correctamente
	\end{itemize}
	%%%%%%%%%%%%%%%%%%%%%%%%%%%%%
	\subsection{Terminado con redireccion}
	\begin{itemize}	
		\item La página web estaba completa, sin embargo redireccionaba a otras paginas como eventos
		o borrar al completar un form o pedir datos como se muestra a continuación.
	\end{itemize}
	%\lstinputlisting[language=JS, caption={Index.js},numbers=left,]{src/indexjs02.js}
	%\lstinputlisting[language=HTMl, caption={Index.html terminado con redireccionamiento y js incrustado},numbers=left,]{src/index02.html}
	\begin{figure}[H]
		\centering
		\includegraphics[width=0.8\textwidth,keepaspectratio]{img/localhost02.png}
	\end{figure}
	\begin{figure}[H]
		\centering
		\includegraphics[width=0.8\textwidth,keepaspectratio]{img/l-eventos.png}
	\end{figure}
	\begin{figure}[H]
		\centering
		\includegraphics[width=0.8\textwidth,keepaspectratio]{img/ver-eventos01.png}
	\end{figure}
	%%%%%%%%%%%%%%%%%%%%%%%%%%%%%
	\subsection{Terminado completamente}
	\begin{itemize}	
		\item Se borraron los console.log() en index.js.
		\item Se externalizo el script del html.
		\item Index.html completamente terminado y con estilo a parte.
	\end{itemize}
	%\lstinputlisting[language=JS, caption={Index.js limpio},numbers=left,]{src/indexjs03.js}
	%\lstinputlisting[language=HTMl, caption={Index.html terminado sin redireccionamiento con CSS y script externos},numbers=left,]{src/index03.html}
	%\lstinputlisting[language=JS, caption={script.js script de funciones para realizar acciondes con el servidor NodeJS},numbers=left,]{src/script.js}
	%\lstinputlisting[language=HTMl, caption={Hoja de estilos },numbers=left,]{src/css01.css}

	\begin{figure}[H]
		\centering
		\includegraphics[width=0.8\textwidth,keepaspectratio]{img/localhost03.png}
	\end{figure}
	\begin{figure}[H]
		\centering
		\includegraphics[width=0.8\textwidth,keepaspectratio]{img/crea-evento.png}
	\end{figure}
	\begin{figure}[H]
		\centering
		\includegraphics[width=0.8\textwidth,keepaspectratio]{img/ver-eventos02.png}
	\end{figure}
	\begin{figure}[H]
		\centering
		\includegraphics[width=0.8\textwidth,keepaspectratio]{img/eliminar.png}
	\end{figure}
	%%%%%%%%%%%%%%%%%%%%%%%%%%%%%
	\subsection{Estructura de laboratorio 01}
	\begin{itemize}	
		\item El contenido que se entrega en este laboratorio es el siguiente:
	\end{itemize}
	
\begin{lstlisting}[style=ascii-tree]
	//se omitieron las subcarpetas y archivos de node_modules al ocupar las de 500 lineas de espacio
	C:\USERS\PAULO\PW2-LAB-C-23A\LAB3
	|	├───express
	|   |   estilo.css
	|   |   index.html
	|   |   index.js
	|   |   package-lock.json
	|   |   package.json
	|   |   script.js
	|   |   
	|   ├───agenda
	|   ├───eventos
	|   ├───node_modules
	|   |   |   *
	|   |
	|   └───priv
	|           poema.txt
	|
	└───latex
		|   lab3_paulo-hidalgo.tex
		|	lab3_paulo-hidalgo.pdf
		|
		├───build
		|       lab3_paulo-hidalgo.aux
		|       lab3_paulo-hidalgo.fdb_latexmk
		|       lab3_paulo-hidalgo.fls
		|       lab3_paulo-hidalgo.log
		|       lab3_paulo-hidalgo.out
		|       lab3_paulo-hidalgo.pdf
		|       lab3_paulo-hidalgo.synctex(busy)
		|
		├───img
		|       crea-evento.png
		|       eliminar.png
		|       l-eventos.png
		|       localhost01.png
		|       localhost02.png
		|       localhost03.png
		|       localhost_agenda.png
		|       logo_abet.png
		|       logo_episunsa.png
		|       logo_unsa.jpg
		|       Segundo-commit.png
		|       ultimo-commit.png
		|       ver-eventos01.png
		|       ver-eventos02.png
		|
		└───src
				css01.css
				index01.html
				index02.html
				index03.html
				indexjs01.js
				indexjs02.js
				indexjs03.js
				script.js
\end{lstlisting}
\subsection{Pregunta: En el Ejemplo Hola Mundo con NodeJS. ¿Qué pasó con la línea: Content type ...?}
\begin{itemize}
	\item No esta debido a que no es necesario ya que devuelve una respuesta en JSON, que después
	es tomada por el cliente e insertada en un div de html ya existente.
\end{itemize}
	\clearpage
	\section{\textcolor{red}{Rúbricas}}

	\subsection{\textcolor{red}{Rúbrica para el contenido del Informe y demostración}}
	\begin{itemize}			
		\item El alumno debe marcar o dejar en blanco en celdas de la columna \textbf{Checklist} si cumplio con el ítem correspondiente.
		\item Si un alumno supera la fecha de entrega,  su calificación será sobre la nota mínima aprobada, siempre y cuando cumpla con todos lo items.
		\item El alumno debe autocalificarse en la columna \textbf{Estudiante} de acuerdo a la siguiente tabla:
	
		\begin{table}[ht]
			\caption{Niveles de desempeño}
			\begin{center}
			\begin{tabular}{ccccc}
    			\hline
    			 & \multicolumn{4}{c}{Nivel}\\
    			\cline{1-5}
    			\textbf{Puntos} & Insatisfactorio 25\%& En Proceso 50\% & Satisfactorio 75\% & Sobresaliente 100\%\\
    			\textbf{2.0}&0.5&1.0&1.5&2.0\\
    			\textbf{4.0}&1.0&2.0&3.0&4.0\\
    		\hline
			\end{tabular}
		\end{center}
	\end{table}	
	
	\end{itemize}
	
	\begin{table}[H]
		\caption{Rúbrica para contenido del Informe y demostración}
		\setlength{\tabcolsep}{0.5em} % for the horizontal padding
		{\renewcommand{\arraystretch}{1.5}% for the vertical padding
		%\begin{center}
		\begin{tabular}{|p{2.7cm}|p{7cm}|x{1.3cm}|p{1.2cm}|p{1.5cm}|p{1.1cm}|}
			\hline
    		\multicolumn{2}{|c|}{Contenido y demostración} & Puntos & Checklist & Estudiante & Profesor\\
			\hline
			\textbf{1. GitHub} & Hay enlace URL activo del directorio para el  laboratorio hacia su repositorio GitHub con código fuente terminado y fácil de revisar. &2 &X &2 & \\ 
			\hline
			\textbf{2. Commits} &  Hay capturas de pantalla de los commits más importantes con sus explicaciones detalladas. (El profesor puede preguntar para refrendar calificación). &4 &X &4 & \\ 
			\hline 
			\textbf{3. Código fuente} &  Hay porciones de código fuente importantes con numeración y explicaciones detalladas de sus funciones. &2 &X &2 & \\ 
			\hline 
			\textbf{4. Ejecución} & Se incluyen ejecuciones/pruebas del código fuente  explicadas gradualmente. &2 &X &2 & \\ 
			\hline			
			\textbf{5. Pregunta} & Se responde con completitud a la pregunta formulada en la tarea.  (El profesor puede preguntar para refrendar calificación).  &2 &X &2 & \\ 
			\hline	
			\textbf{6. Fechas} & Las fechas de modificación del código fuente estan dentro de los plazos de fecha de entrega establecidos. &2 &X &2 & \\ 
			\hline 
			\textbf{7. Ortografía} & El documento no muestra errores ortográficos. &2 &X &2 & \\ 
			\hline 
			\textbf{8. Madurez} & El Informe muestra de manera general una evolución de la madurez del código fuente,  explicaciones puntuales pero precisas y un acabado impecable.   (El profesor puede preguntar para refrendar calificación).  &4 &X &4 & \\ 
			\hline
			\multicolumn{2}{|c|}{\textbf{Total}} &20 & &20 & \\ 
			\hline
		\end{tabular}
		%\end{center}
		%\label{tab:multicol}
		}
	\end{table}
	
\clearpage

\section{Referencias}
\begin{itemize}			
	\item \url{https://github.com/rescobedoq/pw2/tree/main/labs/lab03}
	\item \url{https://www.w3schools.com/nodejs/nodejs_filesystem.asp}
\end{itemize}				
\end{document}